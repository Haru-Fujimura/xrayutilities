% This file is part of xrayutilities.
%
% xrutils is free software; you can redistribute it and/or modify 
% it under the terms of the GNU General Public License as published by 
% the Free Software Foundation; either version 2 of the License, or 
% (at your option) any later version.
%
% This program is distributed in the hope that it will be useful,
% but WITHOUT ANY WARRANTY; without even the implied warranty of
% MERCHANTABILITY or FITNESS FOR A PARTICULAR PURPOSE.  See the
% GNU General Public License for more details.
%
% You should have received a copy of the GNU General Public License
% along with this program; if not, see <http://www.gnu.org/licenses/>.
%
% Copyright (C) 2009 Eugen Wintersberger <eugen.wintersberger@desy.de>
% Copyright (C) 2009-2010 Dominik Kriegner <dominik.kriegner@aol.at>

%%%installation notes

Installing {\tt xrutils} is a three steps process
\begin{enumerate}
\item install required C libraries and Python modules
\item build and install the {\tt xrutils} C library
\item install the Python module
\end{enumerate}
All steps are described in detail in this chapter. The package can be installed on Linux, Mac OS X and Microsoft Windows, however it is only tested on Linux/Unix platforms. Due to the lack of an package manager the installation on MS Windows platforms is cumbersome.

\section{Required third party software}
To keep the coding effort as small as possible {\tt xrutils} depends on a 
large number of third party libraries and Python modules. 

The needed dependencies are: 
\begin{description}
\item[GCC] Gnu Compiler Collection or an compatible C compiler. On windows you most probably should use MinGW or CygWin.
\item[VTK] the Visualization Toolkit, a large C++ library for 3D data
visualization (optionally)
\item[HDF5] a versatile binary data format (library is implemented in C).
Although the library is not called directly, it is needed by the pytables Python
module (see below).
\item[Python] the scripting language in which most of {\tt xrutils} code is written in.
\item[Latex] a typesetting system used to write the documentation of the package (optionally)
\item[subversion] a version control system used to keep track on the {\tt xrutils} development and at the moment the only way to obtain the sources for installation.
\end{description}

Additionally, the following Python modules are needed in order to make 
{\tt xrutils} work as intended
\begin{description}
\item[Scons] a pythonic autotools/make replacement used for building the C library.
\item[setuptools] a python package installer used for the installation of the {\tt xrutils} Python package.
\item[Numpy] a Python module providing numerical array objects
\item[Scipy] a Python module providing standard numerical routines, which is heavily using numpy arrays
\item[VTK Python bindings] usually contained in the VTK distribution, however,
on some system the Python bindings must be installed separately (optionally)
\item[Python-Tables] a powerful Python interface to HDF5. On windows you need to install numexpr as PyTables depends on this package.
\item[Matplotlib] a Python module for high quality 1D and 2D plotting
\item[pyvtk] reading and writing legacy VTK files (optionally)
\item[IPython] although not a dependency of {\tt xrutils} the IPython shell is perfectly suited for the interactive use of the {\tt xrutils} python package.
\end{description}
After installing all required packages you can continue with installing and
building the C library.

\section{Building and installing the C library}

{\tt xrutils} user the SCons build system to compile the C components of the
system. You can build the library simply by typing 
\begin{verbatim}
>scons
\end{verbatim}
in the root directory of the source distribution. To build using debug flags ({\tt -g -O0}) type
\begin{verbatim}
>scons debug=1
\end{verbatim}
instead. After building the library can be installed with
\begin{verbatim}
>scons install --prefix=<install path>
\end{verbatim}
The library is installed in {\tt<install path>/lib}. It is absolutely necessary
to perform this step before installing the Python module since during
installation of the C library a file {\tt config.py} is created in the top
package of the Python module containing the path to the C library. This is in so
far important that the {\tt ctypes} module used to access C library functions
uses this path to find the library. In principal this step could have been
omitted on Linux systems where environment variables can be used to tell the
system where to look for shared libraries. However, on Windows systems, for
instance, no such facility exists and therefore, relying on such mechanisms
would make the installation process highly platform dependent. 

The documentation can be built with 
\begin{verbatim}
>scons doc
\end{verbatim}
which creates this file under {\tt doc/manual/}.

Finally after installing the C library you can continue with installing the
Python module.

\section{Installing the Python package}

Installation of the Python module is done via the {\tt distutils} package
shipped with every Python distribution. Actually, two installation modes are 
supported
\begin{itemize}
 \item system wide installation in the system wide Python installation
 \item local to a single user only.
\end{itemize}

In the first case, to install {\tt xrutils} into the global Python installation
simply run
\begin{verbatim}
>python setup.py install
\end{verbatim}
From the root directory of the {\tt xrutils} source distribution (this is the
directory where also the {\tt SConstruct} file resides). Since your system-wide
Python installation is setup correctly you are done with the installation.

If you want to install
{\tt xrutils} only local to a single user use
\begin{verbatim}
>python setup.py install --home=<local install path>
\end{verbatim}
In this case the {\tt xrutils} package is installed under {\tt <local install
path>/lib/python}. Finally you need to make your Python installation aware of
were to look for the module. For this case append the installation directory to
your {\tt PYTHONPATH} environment variable by 
\begin{verbatim}
>export PYTHONPATH=$PYTHONPATH:<local install path>/lib/python
\end{verbatim}
on a Unix/Linux terminal. Or, to make this configuration persistent append this line to
your local {\tt .bashrc} file in your home directory. On MS Windows you would like to create a environment variable in the 
system preferences under system in the advanced tab.
