%%%documentation of the Experimental classes included in xrutils

The {\tt Experiment} class and derived classes provide routines to help performing X-ray diffraction experiments. This includes methods to calculate the diffraction angles needed to align samples and to convert data between angular and reciprocal space.

\section{HXRD experiments}

Methods for high angle x-ray diffraction experiments. Mostly for experiments performed in coplanar scattering geometry. An example will be given for the calculation of the position of Bragg reflections.

\begin{lstlisting}[caption=calculation of angles for Si Bragg reflections]
import xrutils as xu

Si = xu.materials.Si  # load material from materials submodule

# initialize experimental class with directions from experiment
exp = xu.HXRD(Si.Q(1,1,-2),Si.Q(1,1,1))

# calculate angles and print them to the screen
angs = exp.Q2Ang(Si.Q(1,1,1))
print "Si (111)"
print "om: %8.3f" %angs[1] 
print "tt: %8.3f" %angs[2]

angs = exp.Q2Ang(Si.Q(2,2,4))
print "Si (224)"
print "om: %8.3f" %angs[1] 
print "tt: %8.3f" %angs[2]
\end{lstlisting}


\section{GID experiments}

Implementation of this class is missing

\section{Powder diffraction}

The powder diffraction class is able to convert powder scans from angular to reciprocal space and furthermore powder scans of materials can be simulated in a very easy way.

\begin{lstlisting}[caption=conversion between angular and reciprocal space using the powder diffraction class]
import xrutils as xu

energy = 10000 # eV
powder = xu.Powder(xu.materials.Si,en=energy) # just give some material

# convert absolute q-space value to theta = 2theta/2
theta = powder.Q2Ang(2.0) # 2.0 A^{-1} 
# and back
qpos = powder.Ang2Q(theta) # qpos = 2.0
\end{lstlisting}

\begin{lstlisting}[caption=simulation of an powder diffraction scan and two distinct ways of plotting the results]
import xrutils as xu
import matplotlib.pyplot as plt

energy = (2*8048 + 8028)/3. # copper k alpha 1,2

# creating Indium powder 
In_powder = xu.Powder(xu.materials.Indium,en=energy)
# calculating the reflection strength for the powder
In_powder.PowderIntensity()

# convoluting the peaks with a gaussian in q-space
peak_width = 0.01 # in q-space
resolution = 0.0005 # resolution in q-space
In_th,In_int = In_powder.Convolute(resolution,peak_width)

plt.figure(); plt.clf()
ax1 = plt.subplot(111)
plt.xlabel(r"2Theta (deg)"); plt.ylabel(r"Intensity")
# plot the convoluted signal
plt.plot(In_th*2,In_int/In_int.max(),'k-',label="Indium powder convolution")
# plot each peak in a bar plot
plt.bar(In_powder.ang*2, In_powder.data/In_powder.data.max(), width=0.3, bottom=0, linewidth=0, color='r',align='center', orientation='vertical',label="Indium powder bar plot")

plt.legend(); ax1.set_xlim(15,100); plt.grid()
\end{lstlisting}

 