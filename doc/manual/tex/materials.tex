% This file is part of xrayutilities.
%
% xrayutilities is free software; you can redistribute it and/or modify 
% it under the terms of the GNU General Public License as published by 
% the Free Software Foundation; either version 2 of the License, or 
% (at your option) any later version.
%
% This program is distributed in the hope that it will be useful,
% but WITHOUT ANY WARRANTY; without even the implied warranty of
% MERCHANTABILITY or FITNESS FOR A PARTICULAR PURPOSE.  See the
% GNU General Public License for more details.
%
% You should have received a copy of the GNU General Public License
% along with this program; if not, see <http://www.gnu.org/licenses/>.
%
% Copyright (C) 2009 Eugen Wintersberger <eugen.wintersberger@desy.de>
% Copyright (C) 2010 Dominik Kriegner <dominik.kriegner@aol.at>

%%%documentation of the materials submodule

{\tt xrutils} provides a set of classes to describe crystal lattices and 
materials.

Examples show how to define a new material by definings its lattice and deriving a new material, furthermore materials can be used to calculate the structure factor of a Bragg reflection for an specific energy or the energy dependency of its structure factor for anomalous scattering.

\begin{lstlisting}[caption=defining a new material from scratch. This consists of an lattice with base and the type of atoms with elastic constantsof the material.]
import xrutils as xu

# defining a ZincBlendeLattice with two types of atoms and lattice constant a
def ZincBlendeLattice(aa,ab,a):
    #create lattice base
    lb = xu.materials.LatticeBase()
    lb.append(aa,[0,0,0])
    lb.append(aa,[0.5,0.5,0])
    lb.append(aa,[0.5,0,0.5])
    lb.append(aa,[0,0.5,0.5])
    lb.append(ab,[0.25,0.25,0.25])
    lb.append(ab,[0.75,0.75,0.25])
    lb.append(ab,[0.75,0.25,0.75])
    lb.append(ab,[0.25,0.75,0.75])
    
    #create lattice vectors
    a1 = [a,0,0]
    a2 = [0,a,0]
    a3 = [0,0,a]
    
    l = xu.materials.Lattice(a1,a2,a3,base=lb)    
    return l

# defining InP, no elastic properties are given, 
# helper functions exist to create the (6,6) elastic tensor for cubic materials 
InP  = xu.materials.Material("InP",ZincBlendeLattice(xu.elements.In, xu.elements.P,5.8687), numpy.zeros((6,6),dtype=numpy.double))
# InP is of course already included in the xu.materials module
\end{lstlisting}

\begin{lstlisting}[caption=calculation of the reflection strength of a Bragg reflection]
import xrutils as xu
import numpy

# defining material and experimental setup
InAs = xu.materials.InAs
energy= 8048 # eV

# calculate the structure factor for InAs (111) (222) (333)
hkllist = [[1,1,1],[2,2,2],[3,3,3]]
for hkl in hkllist:
    qvec = InAs.Q(hkl)
    F = InAs.lattice.StructureFactor(energy,qvec)
    print(" |F| = %8.3f" %numpy.abs(F))
\end{lstlisting}

\begin{lstlisting}[caption=energy dependency of the structure factor]
import xrutils as xu
import numpy
import matplotlib.pyplot as plt

# defining material and experimental setup
InAs = xu.materials.InAs
energy= numpy.linspace(500,20000,5000) # 500 - 20000 eV

F = InAs.lattice.StructureFactorForEnergy(energy,InAs.Q(1,1,1))

plt.figure(); plt.clf()
plt.plot(energy,F.real,'k-',label='Re(F)')
plt.plot(energy,F.imag,'r-',label='Imag(F)')
plt.xlabel("Energy (eV)"); plt.ylabel("F"); plt.legend()
\end{lstlisting}

It is also possible to calculate the structure factor of atoms which may be needed for input into XRD simulations.

\begin{lstlisting}[caption=components of the structure factor for simulations]
# f = f0(|Q|) + f1(en) + j * f2(en)
import xrutils as xu
import numpy

Fe = xu.materials.elements.Fe # iron atom
Q = numpy.array([0,0,1.9],dtype=numpy.double)
en = 10000 # energy in eV

print "Iron (Fe): E: %9.1f eV" % en
print "f0: %8.4g" % Fe.f0(numpy.linalg.norm(Q))
print "f1: %8.4g" % Fe.f1(en)
print "f2: %8.4g" % Fe.f2(en)
\end{lstlisting}


